\documentclass[aspectratio=169]{beamer}
\usepackage{graphicx}

\usetheme{default}

\graphicspath{{ .} }

\title{CORE-Storage}
\subtitle{Concepts and uses}
\author{
	Marius Dieckmann\inst{1,2,3}\newline\url{Marius.Dieckmann@computational.bio.uni-giessen.de}
}

\institute[shortinst]{\textsuperscript{1} Justus-Liebig-Universität Gießen \and \inst{2} NFDI4Biodiversity \and \inst{3} de.NBI}

\begin{document}
	\begin{frame}[plain]
		\maketitle
	\end{frame}

	\begin{frame}{NFDI RDC introduction}
		
	\end{frame}

	\begin{frame}{NFDI RDC concept}
		\begin{figure}
			\includegraphics[scale=0.3]{../../images/RDCConcept.png}
		\end{figure}
	\end{frame}

	\begin{frame}{Basic concept}
		\begin{itemize}
			\item API based scalable data management system
			\item Simple data structure with consistent versioning
			\item Versioning schema based on semantic versioning
			\item Object history (no implemented)
			\item gRPC API with pre-generated clients stubs
			\item HTTP-REST gateway with OpenAPI documentation
			\item Update event notification via event streaming
			\item Search engine for JSON based metadata
		\end{itemize}
		
	\end{frame}
	\begin{frame}{NFDI4Biodiversity Multicloud}
		\begin{figure}
			\includegraphics[scale=0.52]{../../images/NFDIMulticloudconcept.png}
		\end{figure}
	\end{frame}
	\begin{frame}{NFDI4Biodiversity Dataflow}
		\begin{figure}
			\includegraphics[scale=0.52]{../../images/DataFlowOdonata.png}
		\end{figure}
	\end{frame}
\end{document}